\section{Specifica del problema}
Si richiede di realizzare un software con interfaccia grafica per sistemi Windows-based che consenta la gestione dei clienti. Il software in questione terrà traccia dei clienti e delle varie commissioni
associate ad essi. Dei clienti ci interessano nome, cognome, il numero di telefono e email, mentre per ogni commissione ci interessano
la data di scadenza e la descrizione.

Il software sarà composto da un menù principale con 4 tab 
\begin{itemize}
        \item \textbf{Home:} Questa è la schermata principale con cui si apre il software, all'interno
        troviamo la lista delle commissioni con scadenza in settimana e la possibilità di visualizzare, modificare
        e aggiungere commissioni. Sempre in questa schermata si ha la possibilità di salvare le modifiche appartate.
        \item \textbf{Rubrica:} In questa schermata troviamo tutti i contatti salvati con la possibilità di aggiungerne nuovi, 
        modificare quelli già presenti o eliminarli (eliminando tutte le relative commissioni).
        \item \textbf{Commissioni:} Qui possiamo visualizzare tutte le commissioni (completate e non). Anche qui sono presenti i tasti 
        visualizza, modifica e elimina.
        \item \textbf{Scadenze:} In quest'ultima tab verranno mostrate solo le commissioni non ancora completate, con la possibilità di visualizzarle.

\end{itemize}


\medskip
Il software inoltre offre la possibilità di salvare i dati in locale e anche in formato Json per un eventuale
salvataggio su database.


% Una volta compilati i campi dedicati all'inserimento di un nuovo cliente o di una nuova commissione,
% Potremmo visualizzare 


\medskip
Ogni elemento, cliente o commissione che sia, può essere visualizzato aggiunto, modificato o eliminato.
Una volta selezionato un elemento.
Quando si richiede la modifica, verrà mostrata una schermata dedicata, mentre per quanto concerne l'eliminazione  
sarà richiesta una conferma prima di eliminare definitivamente l'elemento.

\medskip
Una volta conclusa la sessione, prima della chiusura del programma, se sono stati apportati cambiamenti non salvati su file,
il programma notifcherà l'utente. 