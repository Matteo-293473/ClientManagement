\section{Specifica del problema}
Si richiede di realizzare un videogioco di tipo musicale ispirato all saga di successo \hyperlink{https://it.wikipedia.org/wiki/Guitar_Hero_(serie)}{Guitar Hero} in cui l'utente impersona un musicista.
Tale videogioco avrà però come strumento utilizzato solamente il rullante (il principale strumento percussivo in una batteria).
Tale scelta è guidata dal fatto che il videogioco, oltre a sfidare l'utente così da renderne competitivo l'utilizzo, potrà essere utilizzato per imparare e migliorare la propria tecnica sullo strumento.

\medskip
Il videogioco sarà composto da una schermata iniziale in cui l'utente potrà decidere se iniziare una nuova partita, visualizzare i risultati ottenuti dagli utenti del videogioco o chiudere il programma.

\medskip
Nel caso in cui l'utente voglia iniziare una nuova partita, verrà reindirizzato nell'opportuna schermata dove verrà richiesto l'inserimento del suo nome (necessario per poter poi memorizzare il punteggio totalizzato della partita), la modalità alla quale si desidera giocare e la velocità di esecuzione iniziale (espressa in \emph{BPM}, battiti per minuto).

\medskip
Le modalità implementate nel videogioco sono due:
\begin{itemize}
        \item nella prima modalità, chiamata \emph{combinazioni casuali}, le note sono generate, molto intuitivamente, in maniera casuale, senza nessuna forzatura sul numero di note standard o speciali o sulla loro posizione;

        \item nella seconda modalità, chiamata \emph{mani alternate}, le note generate sono alternate, ovvero se una nota è destra, la successiva sarà sinistra, e così via.
\end{itemize}

\medskip
Una volta compilati opportunamente i campi della schermata di una nuova partita, verrà mostrata la schermata di gioco con la quale l'utente interagirà.
Essa sarà composta da un rullante e due bacchette inizialmente alzate. Alla pressione del tasto \textbf{c} e \textbf{n} le bacchette (rispettivamente sinistra e destra) reagiranno all'input dell'utente colpendo il rullante.
La sequenza di colpi da effettuare verrà rappresentata da immagini rappresentanti note che si muovono lungo due linee che congiungono i due angoli superiori dello schermo sino al rullante.
Nel momento in cui il punto raggiunge il rullante e in contemporanea viene effettuato il colpo allora esso verrà conteggiato come corretto.

\medskip
Il videogioco conterrà un semplice sistema di calcolo del punteggio totalizzato dall'utente.
Tale punteggio sarà attribuito in base alla precisione del colpo.

Nel caso l'utente colpisca il rullante esattamente nell'istante in cui viene richiesto allora aumenterà il proprio punteggio di $100$ punti (colpo perfetto).
Nel caso in cui il colpo non sia precisamente nella posizione segnalata (colpo standard), ogni $2 ms$ di ritardo o di anticipo (rispetto al colpo perfetto) risulterà in una penalità di $1$ punto alla quantità che verrà aggiunta al punteggio totale.
Nel caso l'utente colpisca il rullante quando il colpo non è ancora giunto sul rullante (colpo errato) non verrà attribuito alcun punteggio, dopo $20$ errori di questo tipo la partita termina.

Durante la partita apparirà, in maniera casuale, al posto di un colpo standard un colpo speciale identificato diversamente dagli altri. Il punteggio assegnato nel caso in cui tale colpo venga eseguito in maniera corretta sarà doppio.

La difficoltà del gioco è determinata dal numero di BPM che si sta affrontando. Essi aumentano progressivamente ogni 5 colpi corretti (perfetti o standard) effettuati.

A fine partita il punteggio verrà mostrato all'utente e inserito in un file \texttt{csv} contenente tutti i punteggi effettuati con il gioco. Tali punteggi saranno visualizzabili a partire dalla schermata principale tramite un'apposita schermata raggiungibile premendo il tasto dedicato.