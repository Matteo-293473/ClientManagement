\section{Specifica del problema}
Si richiede di realizzare un software con interfaccia grafica per sistemi Windows-based che consenta la gestione dei clienti. Il software in questione terrà traccia dei clienti e delle varie commissioni
associate ad essi. Dei clienti ci interessano nome, cognome, il numero di telefono e email, mentre per ogni commissione ci interessano
la data di scadenza e la descrizione.

Il software sarà composto da un menù principale con 4 tab 
\begin{itemize}
        \item \textbf{Home:} Questa è la schermata principale con cui si apre il software, all'interno
        troviamo la lista delle commissioni con scadenza in settimana e la possibilità di modificare
        e aggiungere commissioni.
        \item \textbf{Rubrica:} In questa schermata troviamo tutti i contatti %aggiungi la possibilià di selezionare un contatto e aggiungere una commissione veloce
        salvati con la possibilità di aggiungerne nuovi o di modificare quelli già presenti. %possibilità di mandare un'email al volo?% 
        \item \textbf{Commissioni:} Qui possiamo visualizzare tutte le commissioni (completate e non)
        \item \textbf{Scadenze:} In quest'ultima tab verranno mostrate solo le commissioni non ancora completate %??%
\end{itemize}


\medskip
Il software inoltre offre la possibilità di salvare i dati in locale e anche in formato Json per un eventuale
salvataggio su database.



Una volta compilati opportunamente i campi della schermata di una nuova partita, verrà mostrata la schermata di gioco con la quale l'utente interagirà.
Essa sarà composta da un rullante e due bacchette inizialmente alzate. Alla pressione del tasto \textbf{c} e \textbf{n} le bacchette (rispettivamente sinistra e destra) reagiranno all'input dell'utente colpendo il rullante.
La sequenza di colpi da effettuare verrà rappresentata da immagini rappresentanti note che si muovono lungo due linee che congiungono i due angoli superiori dello schermo sino al rullante.
Nel momento in cui il punto raggiunge il rullante e in contemporanea viene effettuato il colpo allora esso verrà conteggiato come corretto.

\medskip
Il videogioco conterrà un semplice sistema di calcolo del punteggio totalizzato dall'utente.
Tale punteggio sarà attribuito in base alla precisione del colpo.

Nel caso l'utente colpisca il rullante esattamente nell'istante in cui viene richiesto allora aumenterà il proprio punteggio di $100$ punti (colpo perfetto).
Nel caso in cui il colpo non sia precisamente nella posizione segnalata (colpo standard), ogni $2 ms$ di ritardo o di anticipo (rispetto al colpo perfetto) risulterà in una penalità di $1$ punto alla quantità che verrà aggiunta al punteggio totale.
Nel caso l'utente colpisca il rullante quando il colpo non è ancora giunto sul rullante (colpo errato) non verrà attribuito alcun punteggio, dopo $20$ errori di questo tipo la partita termina.

Durante la partita apparirà, in maniera casuale, al posto di un colpo standard un colpo speciale identificato diversamente dagli altri. Il punteggio assegnato nel caso in cui tale colpo venga eseguito in maniera corretta sarà doppio.

La difficoltà del gioco è determinata dal numero di BPM che si sta affrontando. Essi aumentano progressivamente ogni 5 colpi corretti (perfetti o standard) effettuati.

A fine partita il punteggio verrà mostrato all'utente e inserito in un file \texttt{csv} contenente tutti i punteggi effettuati con il gioco. Tali punteggi saranno visualizzabili a partire dalla schermata principale tramite un'apposita schermata raggiungibile premendo il tasto dedicato.