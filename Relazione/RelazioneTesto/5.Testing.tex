\section{Testing}
        Lo sviluppo del software è stato per gran parte effettuato utilizzando la tecnica del \emph{pair-programming}.
        Tale tecnica ci ha permesso di analizzare ogni scelta implementativa a fondo senza effettuare nessuna scelta in modo approssimativo; il risultato è un software la cui probabilità di presenza di errori è piuttosto bassa.
        
        Il testing di tale software è stato effettuato inizialmente in modalità \texttt{whitebox} durante la fase di sviluppo, testando ogni condizione limite che si sarebbe potuta presentare.
        Ciò ha permesso di arginare in fase iniziale la maggior parte degli errori che sarebbero potuti emergere a lavoro completato.
        
        \vspace{0.3cm}
        In seguito il software, giunti a quella che è la versione finale, è stato testato con le modalità \texttt{blackbox}.
        Nella sottosezioni seguenti verranno mostrati vari \emph{screenshot} del software corrispondenti ai vari casi d'uso identificati nella sezione corrispondente.
    
        \newpage
        \subsection{Inizio di una partita}
            \begin{figure}[h!]
               % \includegraphics[width=\linewidth]{testing/menu_iniziale.PNG}
            \end{figure}
            Cliccando sul pulsante \emph{Nuova partita} viene mostrata la schermata seguente
            \begin{figure}[h!]
               % \includegraphics[width=\linewidth]{testing/nuova_partita.PNG}
            \end{figure}
        
        \newpage
        \subsection{Visualizzazione punteggi}
            Nel caso l'utente abbia già giocato almeno una partita vengono mostrati i punteggi totalizzati.
            \begin{figure}[h!]
               % \includegraphics[width=\linewidth]{testing/record_avaiable.PNG}
            \end{figure}
            
            Nel caso invece l'utente non abbia ancora giocato nessuna partita viene mostrato un messaggio che comunica l'assenza di punteggi registrati.
            \begin{figure}[h!]
               % \includegraphics[width=\linewidth]{testing/record_not_avaiable.PNG}
            \end{figure}
            
        \newpage
        \subsection{Impostazione di nome, modalità di gioco e BPM iniziali}
            \begin{figure}[h!]
               % \includegraphics[width=\linewidth]{testing/nuova_partita.PNG}
            \end{figure}
            
            Nel caso l'utente tenti di procedere senza aver inserito il proprio nome la partita non viene iniziata.
            \begin{figure}[h!]
               % \includegraphics[width=\linewidth]{testing/name_not_set.PNG}
            \end{figure}
            
        \newpage
        \subsection{Effettuazione colpo}
            \begin{figure}[h!]
               % \includegraphics[width=\linewidth]{testing/hit_before_end.PNG}
            \end{figure}
        
        \subsection{Fine della partita}
            \begin{figure}[h!]
               % \includegraphics[width=\linewidth]{testing/end_game.PNG}
            \end{figure}
        
        \newpage    
        \subsection{Partita in pausa}
            \begin{figure}[h!]
               % \includegraphics[width=\linewidth]{testing/menu_pausa.PNG}
            \end{figure}
        
        \subsection{Partita ripresa da una pausa}
            \begin{figure}[h!]
               % \includegraphics[width=\linewidth]{testing/partita_ripresa.PNG}
            \end{figure}
            
        \newpage
        \subsection{Partita abbandonata da una pausa}
            \begin{figure}[h!]
               % \includegraphics[width=\linewidth]{testing/abbandona_partita.PNG}
            \end{figure}
    