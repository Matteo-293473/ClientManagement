\section{Compilazione ed esecuzione}
    \vspace{1cm}
    Gli strumenti utilizzati per la compilazione del programma sono i seguenti:
    \vspace{0.3cm}
    \begin{itemize}
        \item \textbf{Ambiente di sviluppo}: Visual Studio Community 2017
        \item \textbf{Versione}: 15.9.28307.905
        \item \textbf{Framework}: .NET Framework 4.5.2
    \end{itemize}
    \vspace{1cm}
    Per la compilazione della soluzione, è necessario recarsi nel seguente path dell'ambiente di sviluppo: nella barra dei menù selezionare \emph{Compila \textgreater  Compila soluzione}. In alternativa, se non si compila manualmente la soluzione, all'avvio del programma (tramite apposita icona su Visual Studio), la compilazione avviene in maniera automatica.
    \vspace{0.3cm}
    
    Per eseguire l’applicazione si può utilizzar l'ambiente di sviluppo oppure ricorrere all'eseguibile presente nella cartella del progetto al percorso \texttt{./MasterDrums/MasterDrums/bin/Debug/MasterDrums.exe}.
    
    \vspace{1cm}
    
    I requisiti minimi per per l'esecuzione del programma sono i seguenti:
    \begin{itemize}
        \item Sistema operativo: Windows 7 o successivi
        \item Architettura: 32 o 64 bit
        \item Framework: .NET Framework 4.0
    \end{itemize}
    
    \vspace{0.5cm}
    Non vi sono requisiti di performance particolari, tuttavia si rimanda a consultare i requisiti minimi per il .NET Framework al sito \hyperlink{https://docs.microsoft.com/dotnet/framework/get-started/system-requirements}{Microsoft} tenendo in considerazione che il programma utilizza al più 60 MB di memoria RAM.
    
    \vspace{1cm}
    Il software è stato testato su un computer con la seguente scheda tecnica:
    \begin{itemize}
        \item CPU: Intel(R) Core(TM) i5-8250U CPU @ 1.80 GHz 
        \item RAM: 8GB DDR3
        \item GPU: Intel UHD Graphics 620
        \item SO: Windows 10 Home, Build 1903
        \item Architettura: 64 bit
    \end{itemize}